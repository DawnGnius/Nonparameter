\begin{problem}[17.1]
    The objective in this exercise is to estimate the production function for China’s nongovernmental businesses for the Year 2003. 
    The data include 2052 valid observations on $7$ variables: Output, capital, Labor, Province, Ownership, Industry, and List,  where the first three variables are self-defined, and the other variables are categorical variables (for example, List = 1 if a business is a listed company, 0 otherwise). 
    Please use the dataset (Business03.txt) to answer the following questions:
    
    (1) Run a multiple linear regression (MLR) model by regressing ln(Output) on a constant term, ln(Capital) and ln(Labor) :
    \begin{equation*}
        Y = m(X_1, X_2) + u
    \end{equation*}
    where $Y=ln(Output)$, $X_1=ln(Capital)$, and $X2=ln(Labor)$, and $u$ is the disturbance term. Report the regression results in the standard format. That is, you need to report the regression model, the $t$-values, or the $p$-values or the corresponding standard errors for the coefficients in the model, $R^2$, $\bar{R}^2$ (Adjusted $R^2$), and the $F$ test statistic or its corresponding $p$-value.

    (2) Run a nonparametric regression model by regressing $ln(Output)$ on $ln(Labor)$ and $ln(Labor)$ by using the local constant procedure
    \begin{equation*}
        Y = m(X_1, X_2) + u
    \end{equation*}

    Denote the regression estimates by $\hat{m}_{lc}(x_1,x_2).$. Calculate the $R^2$ based upon the formula
    \begin{equation*}
        R^2=\frac{[\sum_{i=1}^n(Y_i-\bar{Y})(\hat{Y}_i-\bar{\hat{Y}}]^2}{\sum_{i=1}^n(Y_i-\bar{Y})^2\sum_{i=1}^n(\hat{Y}_i-\bar{\hat{Y}})^2},
    \end{equation*}
    which is the squre of the sample correlation between $Y_i$ and $\hat{Y}_i$, where $\hat{Y}_i=\hat{m}_{lc}(X_{1i},X_{2i})$ is the insample predict value for $Y_i$,
    $\bar{Y}=\frac1n\sum_{i=1}^nY_i$, and $\bar{\hat{Y}}=\frac1n\sum_{i=1}^n\hat{Y}_i$. 
    Plot $\hat{Y}$ against $X_1$ and $X_2$ in a three-dimensional diagrm.
    Does the diagram lend any support to the MLR model in part (1)?

    (3) Repeat part (2) by using the local linear procedure. 

    (4) Test the correct specification of the model in part (1) using \emph{np package}.
\end{problem}

\begin{solution}
    See RMarkdown file for detial. 
\end{solution}