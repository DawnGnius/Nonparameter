\begin{problem}
考虑bootstrap包里的scor考试成绩数据. 此数据是88名考生5门课程的考试成绩:mechanics, vectors, algebra, analysis, statistics. 前两门为闭卷考试, 其余
三门为开卷考试.在scor数据下, 记协方差矩阵为$\Sigma$, 其特征根为$\lambda_1>\cdots>\lambda_5>0$. 则
\begin{equation*}
    \theta=\frac{\lambda_1}{\sum_{i=1}^5\lambda_i}
\end{equation*}
表示了主成分中第一主成分对方差的解释比例. 现在记$\theta=\frac{\lambda_1}{\sum_{i=1}^5\lambda_i}$
为样本协方差矩阵$\hat\Sigma$的特征根.

(1). 分别使用Bootstrap方法和Jackknife方法估计$\theta$的估计
\begin{equation*}
    \hat\theta=\frac{\hat\lambda_1}{\sum_{i=1}^5\hat\lambda_i}
\end{equation*}
的偏差和标准差。

(2). 计算在(1)中的Bootstrap重复样本下$\theta$的$95\%$百分位数置信区间和BCa置信区间.
\end{problem}

\begin{solution}
(1)
使用 R 中的 boostrap 包, 及如下代码计算 $\hat{\theta}$ 的偏差和标准误差. 

\begin{lstlisting}[language=R] 
n  <-  nrow(scor)
cov.hat  <-  cov(scor)
eigenvalue.hat  <-  eigen(cov.hat)$values
theta.hat  <-  eigenvalue.hat[1]/sum(eigenvalue.hat) 

B  <-  50
theta.b  <-  numeric(B)
for (i in 1:B) {
    i   <-   sample(1:n, size = n, replace = TRUE)
    cov.b  <-  cov(scor[i,])
    eigenvalue.b  <-  eigen(cov.b)$values
    theta.b[i]  <-  eigenvalue.b[1]/sum(eigenvalue.b)
}
bias.b  <-  mean(theta.b-theta.hat)
se.b  <-  sd(theta.b)
print(bias.b)
print(se.b)
\end{lstlisting} 

得到 $bias()\hat{\theta})$ 以及 $se(\hat{\theta})$ 分别为 $0.05944855$ 和 $0.04929686$.

使用 Jackknife 函数来估计 $\hat{\theta}$ 的偏差和标准误差, 代码如下
\begin{lstlisting}[language=R] 
theta.jack <- numeric(n)
for(i in 1:n){
  cov.jack <- cov(scor[-i,])
  eigenvalue.jack <- eigen(cov.jack)$values
  theta.jack[i] <- eigenvalue.jack[1]/sum(eigenvalue.jack)
}
bias.jack <- (n-1)*(mean(theta.jack)-theta.hat)
print(bias.jack)
se.jack <- sqrt((n-1) *mean((theta.jack - mean(theta.jack))^2))
print(se.jack)
\end{lstlisting}
得到 $bias()\hat{\theta})$ 以及 $se(\hat{\theta})$ 分别为 $0.001069139$ 和 $0.04955231$.

(2) 使用 bootstrap 和 boot 包, 运行如下代码
\begin{lstlisting}[language=R] 
data(scor,package="bootstrap")
theta.boot<-function(scor,i){
    eigenvalue<-eigen(cov(scor[i,],scor[i,]))$values
    eigenvalue[1]/sum(eigenvalue)
}
boot.obj<-boot(scor, theta.boot,R=2000)
print(boot.ci(boot.obj,type = c("perc","bca")))
\end{lstlisting}
得到 $\theta$的$95\%$百分位数置信区间和BCa置信区间分别为 $( 0.5219,  0.7068 ) ,\ ( 0.5219,  0.7068 )$.
\end{solution}

\begin{problem}
设$X_1,\ldots,X_n$表示不同的样本值, $X_1^*,\ldots,X_n^*$为一个Bootstrap抽样, 令$X_1^*,\ldots,X_n^*$, 求期望 $E(\bar{X}^*|X_1,\ldots,X_n)$, $Var(\bar{X}^*|X_1,\ldots,X_n)$以及$E(\bar{X}^*)$和$Var(\bar{X}^*)$.
\end{problem}

\begin{solution}
    \begin{equation*}
        E\left[\overline{X}^{*} | X_{1}, \cdots, X_{n}\right]=E\left[\frac{1}{n} \sum_{i=1}^{n} X_{i}^{*} | X_{1}, \cdots, X_{n}\right]=E\left[X_{1}^{*} | X_{1}, \cdots, X_{n}\right]=\frac{1}{n} \sum_{i=1}^{n} X_{i}
    \end{equation*}

    \begin{equation*}
    \begin{aligned} E\left[\left(\overline{X}^{*}\right)^{2} | X_{1}, \cdots, X_{n}\right] &=\frac{1}{n^2} E\left[\left(\sum_{i=1}^{n} X_{i}^{\star}\right)^{2} | X_{1}, \cdots, X_{n}\right] \\ &=\frac{1}{n^2} \sum_{i=1}^{n} E\left[\left(X_{i}^{\star}\right)^{2} | X_{1}, \cdots, X_{n}\right] \\ &=\frac{1}{n} E\left[\left(X_{1}^{*}\right)^{2} | X_{1}, \cdots, X_{n}\right] \\ &=\frac{1}{n^2} \sum_{i=1}^{n} X_{i}^{2} \end{aligned}
    \end{equation*}

    \begin{equation*}
    \begin{aligned} \operatorname{var}\left(\overline{X}^{\star} | X_{1}, \cdots, X_{n}\right) &=E\left[\left(\overline{X}^{\star}\right)^{2} | X_{1}, \cdots, X_{n}\right]-\left(E\left[\overline{X}^{\star} | X_{1}, \cdots, X_{n}\right]\right)^{2} \\ &=\frac{1}{n^2} \sum_{i=1}^{n} X_{i}^{2}-\left(\frac{1}{n} \sum_{i=1}^{n} X_{i}\right)^{2} \end{aligned}
    \end{equation*}

    再由双期望公式得

    \begin{equation*}
        E\left[\overline{X}^{*}\right]=E\left[\frac{1}{n} \sum_{i=1}^{n} X_{i}\right]=E X
    \end{equation*}

    \begin{equation*}
    \begin{aligned} \operatorname{var}\left(\overline{X}^{*}\right) &=E\left[\operatorname{var}\left(\overline{X}^{*} | X_{1}, \cdots, X_{n}\right)\right]+\operatorname{var}\left(E\left[\overline{X}^{*} | X_{1}, \cdots, X_{n}\right]\right) \\ &=E\left[\frac{1}{n^2} \sum_{i=1}^{n} X_{i}^{2}-\left(\frac{1}{n} \sum_{i=1}^{n} X_{i}\right)^{2}\right]+\operatorname{var}\left(\frac{1}{n} \sum_{i=1}^{n} X_{i}\right) \\ &=\frac{1}{n^2} \sum_{i=1}^{n} E\left[X_{i}^{2}\right]-\frac{1}{n^2} E\left[\left(\sum_{i=1}^{n} X_{i}\right)^{2}\right]+\frac{1}{n^2} \operatorname{var}\left(\sum_{i=1}^{n} X_{i}\right) \\ &=\operatorname{var}(X) / n \end{aligned}
    \end{equation*}
\end{solution}