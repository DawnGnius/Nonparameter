\begin{problem}
    Let $X,X_1,\ldots,X_n i.i.d\sim F$, find kernel $h(x_1,x_2,x_3)$ such that $E_Fh(X_1,X_2,X_3)=E(X-E_FX)^3$
\end{problem}

\begin{solution}
    Let's consider the following equations
    \begin{equation*}
        T=E(X-E X)^{3}=E X^{3}-3 E X E X^{2}+2(E X)^{3}.
    \end{equation*}
    So
    \begin{equation*}
        E(X_1 + X_2 + X_3)^3 = 3EX^3 + 18EXEX^2 + 6(EX)^3.
    \end{equation*}
    Then
    \begin{equation*}
        \begin{split} 
            T &=-\frac{1}{6} E\left(X_{1}+X_{2}+X_{3}\right)^{3}+\frac{3}{2} E X^{3}+3(E X)^{3} \\ 
            &=E\left(-\frac{1}{6}\left(X_{1}+X_{2}+X_{3}\right)^{3}+\frac{1}{2}\left(X_{1}^{3}+X_{2}^{3}+X_{3}^{3}\right)+3\left(X_{1} X_{2} X_{3}\right)\right) 
        \end{split}
    \end{equation*}

    Thus
    \begin{equation*}
        h\left(x_{1}, x_{2}, x_{3}\right)=-\frac{1}{6}\left(x_{1}+x_{2}+x_{3}\right)^{3}+\frac{1}{2}\left(x_{1}^{3}+x_{2}^{3}+x_{3}^{3}\right)+3\left(x_{1} x_{2} x_{3}\right)
    \end{equation*}
    % \begin{equation*}
    %     \begin{split} 
    %         s_n^3 
    %         & = \frac{n}{(n-1)(n-2)} \sum_{i=1}^n (X_i - \bar{X}_n)^3 \\
    %         & = - \frac{1}{6n(n-1)(n-2)} \sum_{i=1}^{n} \sum_{j=1}^{n} \sum_{k=1}^{n} \left(\left(X_{i}-\bar{X}_{n}\right)^{3} + \left(X_{j}-\bar{X}_{n}\right)^{3} - 8\left(X_{k}-\bar{X}_{n}\right)^{3}\right) \\ 
    %         & = - \frac{1}{6n(n-1)(n-2)} \sum_{i=1}^{n} \sum_{j=1}^{n} \sum_{k=1}^{n} \left(\left(X_{i}-\bar{X}_{n}\right) + \left(X_{j}-\bar{X}_{n}\right) - 2\left(X_{k}-\bar{X}_{n}\right) \right)^{3} \\ 
    %         & = - \frac{1}{6n(n-1)(n-2)} \sum_{i=1}^{n} \sum_{j=1}^{n} \sum_{k=1}^{n}  \left(X_{i}+X_{j} - 2X_k \right)^{3} \\ 
    %         & = \frac{1}{\left(
    %             \begin{matrix}
    %                 {n} \\ 
    %                 {3}
    %             \end{matrix}\right)} 
    %             \sum_{i,j,k} \frac{1}{36}\left(2X_{i}-X_{j}-X_k\right)^{3} 
    %     \end{split}
    % \end{equation*}

    % Thus, $h(x_1,x_2,x_3)= \frac{1}{36}\left( 2x_1 - x_2 - x_3 \right)^{3}$

\end{solution}



\begin{problem}
    Prove the $\zeta_1=1/9$ in slide (page 25).
\end{problem}

\begin{solution}
    Let $X\sim F(x), Y\sim G(y)$. Because $X$ and $Y$ are independent, $\tau = 0$. 
    Thus we have
    \begin{equation*}
        \begin{split}
            \zeta_1 & = Cov (h(P_1, P_2), h(P_1, P_3)) \\
            & = E[h(P_1, P_2) h(P_1, P_3)] \\
            & = 1\times P(h(P_1, P_2) h(P_1, P_3)=1) + (-1) \times P(h(P_1, P_2) h(P_1, P_3)=-1) \\
            & = 2P(h(P_1, P_2) h(P_1, P_3)=1)-1
        \end{split}
    \end{equation*}

    Denote $P_i = (X_i,Y_i), i=1,2,3$ , we have
    \begin{equation*}
        \begin{split}
            &  P(h(P_1, P_2) h(P_1, P_3)=1   |P_1, h(P_1, P_2)=1) \\
            = & P(h(P_1, P_3)=1|P_1) \\
            = & P(X_3>X_1, Y_3>Y_1 \text{ or } X_3<X_1, Y_3<Y_1) \\
            = & (1-F(X_1))(1-G(Y_1)) + F(X_1)G(Y_1)
        \end{split}
    \end{equation*}
    and
    \begin{equation*}
        \begin{split}
            &  P(h(P_1, P_2) h(P_1, P_3)=1   | P_1, h(P_1, P_2)=-1) \\
            = & P(h(P_1, P_3)=-1 | P_1) \\
            = & P(X_3<X_1, Y_3>Y_1 \text{ or } X_3>X_1, Y_3<Y_1) \\
            = & F(X_1)(1-G(Y_1)) + (1-F(X_1))G(Y_1)
        \end{split}
    \end{equation*}
    and
    \begin{equation*}
        \begin{split}
            & P(h(P_1, P_2) h(P_1, P_3)=1 |  P_1) \\
            &= P(h(P_1, P_2)=1, h(P_1, P_3)=1 |  P_1) + P(h(P_1, P_2)=-1, h(P_1, P_3)=-1 |  P_1) .
        \end{split}
    \end{equation*}
    Thus
    \begin{equation*}
        \begin{split}
            & P(h(P_1, P_2) h(P_1, P_3)=1) \\
            = & E[P(h(P_1, P_2) h(P_1, P_3)=1 |  P_1)]\\
            = & \int\int [(1-F(X_1))(1-G(Y_1)) + F(X_1)G(Y_1)]^2 + \\
            & \quad\quad\quad\quad [F(X_1)(1-G(Y_1)) + (1-F(X_1))G(Y_1)]^2 d F d G \\
            = & \frac{5}{9}
        \end{split}
    \end{equation*}

    So, we have $\zeta_1=1/9$.
\end{solution}



\begin{problem}
    Let $X_1,\ldots,X_n i.i.d \sim \text{Uniform}(0,\tau)$ kernel $h(x,y)=|x-y|$ and U-statistics $G_n=\frac{1}{{n\choose 2}}\sum_{i<j}|X_i-X_j|$, find the limit distribution of $G_n$.
\end{problem}

\begin{solution}
    $G_n$ is a U-statistic of order $r=2$ with kernel $h(x_1,x_2)=|x_1-x_2|$ can also an unbiased estimate of $E|X_1 - X_2|$ \citep{Dasgupta2008Asymptotic}.

    From the CLT for U-statistics, it follows that
    \begin{equation*}
        \sqrt{n}(G_n -E|X_1 - X_2|) \leadsto N(0, r^2 \zeta_1) 
    \end{equation*}
    where $\zeta_1$ can be calculated as following. %to be $\zeta_1 = \textbf{Var}_F \left( 2X F(X) - X - 2\int_{-\infty}^X y dF(y) \right)$.

    We have $F(x)=x/\tau, \, x\in (0,\tau)$ because $X\sim \text{Uniform}(0,\tau)$. 
    Thus
    \begin{equation*}
        \begin{split} E h(x, X_2) &=E|x-X_2| \\ 
            &=\int_{0}^{\tau}|x-y| \frac{1}{\tau} d y \\ 
            &=\frac{1}{\tau}\left(\int_{0}^{x} x-y d y+\int_{x}^{\tau} y-x d y\right) \\ 
            &=\frac{x^{2}}{\tau}-x+\frac{\tau}{2} ,
        \end{split}
    \end{equation*}
    and
    \begin{equation*}
        \begin{split} 
            E h(X_1, X_2)
            &=E[E[h(X_1, X_2) | X_1]] \\ 
            &=E\left[\frac{X_1^{2}}{\tau}-X_1+\frac{\tau}{2}\right] \\ 
            &=\int_{0}^{\tau} \frac{1}{\tau}\left(\frac{x^{2}}{\tau}-x+\frac{\tau}{2}\right) d x \\ 
            &=\frac{\tau}{3} ,
        \end{split}
    \end{equation*}
    and
    \begin{equation*}
        h_{1}(x) = E h(x, Y)-T=\frac{x^{2}}{\tau}-x+\frac{\tau}{6}
    \end{equation*}
    and 
    \begin{equation*}
        \begin{split} 
            \zeta_{1} &=E\left[h_{1}^{2}(x)\right] \\
            &=E\left(\frac{x^{2}}{\tau}-x+\frac{\tau}{6}\right)^{2} \\ 
            &=\frac{\tau^{2}}{180}.
        \end{split}
    \end{equation*}   
    % \begin{equation*}
    %     \begin{split}
    %         \zeta_1 
    %         & = \textbf{Var}_F \left( 2X^2 / \tau - X - 2\int_0^X y dF(y) \right) \\
    %         & = \textbf{Var}_F \left( 2X^2 / \tau - X - X^2/ \tau \right) \\
    %         & = \textbf{Var}_F \left( X^2 / \tau - X \right) \\
    %         & = \frac{1}{\tau^2} \textbf{Var}_F(X^2) +  \textbf{Var}_F(X) \\
    %         & = \frac{1}{\tau^2} \frac{4\tau^4}{45} + \frac{\tau^2}{12} \\
    %         & = \frac{31}{180} \tau^2
    %     \end{split}
    % \end{equation*}

    Thus,
    \begin{equation*}
        \sqrt{n}\left(G_{n}-\frac{\tau}{3}\right) \rightarrow N\left(0, \frac{\tau^{2}}{45}\right)
    \end{equation*}
\end{solution}